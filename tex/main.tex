\documentclass[12pt]{scrartcl}

\usepackage[english]{babel}
\usepackage[sorting=none,backend=biber]{biblatex}
\usepackage{amsmath}
\usepackage{graphicx}
\usepackage[noend]{algpseudocode}
\usepackage{algorithm}
\usepackage{hyperref}
\usepackage[backend=cairo, outputdir=diagrams]{diagrams-latex}
\addbibresource{ref/ref.bib}

\newtheorem{dfn}{Definition}

\title{BCMtools}
\author{Kai Zhang}
\date{}

\begin{document}

\maketitle

The advent of high resolution 3D genome map provides us with unprecedent
opportunity to examine the high order chromatin in detail. In the mean time,
howeer, it
also raises several computational challenges. For example, a 5KB
resolution contact map for human chromosme 1 can take up to 10GB space on hard
drive. And common tools that have been used to visualize HiC contact map are not
capable of handling huge data like this. Here we present a conpact binary format
that are designed to store large matrix files which is space efficient and
readily accessible. We also provide a library called BCMtools that can create,
process and visualize this binary file. One key feature of BCMtools is that it
uses constant memory to process contact maps regardless it size. Every
processing step is carried out in a streamming fasion. This is
achieved by storing and processing the data right on disk without even reading
them into memory.

\subsection{BCM format}

BCM file consists of a header section followed by the data section. 

\subsection{Features}

\subsection{Implementation}



\printbibliography

\end{document}
